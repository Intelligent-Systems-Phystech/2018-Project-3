\documentclass[12pt,twoside]{article}
    \usepackage{jmlda}
    %\NOREVIEWERNOTES
    \title
        {Динамическое выравнивание многомерных временных рядов}
    \author
        {Гончаров~А.\,В., Моргачев~Г.\,И., Смирнов~В.\,, Липницкая~Т.\,} % основной список авторов, выводимый в оглавление
    \thanks{
        Работа выполнена при финансовой поддержке РФФИ, проект \No\,00-00-00000.
        Научный руководитель:  Гончаров~А.\,В.
        Задачу поставил:  Гончаров~А.\,В.
        Консультант:  Гончаров~А.\,В.
    }
    \email
        {morgachev.gi@phystech.edu, smirnov.vs@phystech.edu, tanya.lipnizky@yandex.ru}
    \organization
        {МФТИ}

    \abstract{
        Данная работа посвящена задаче получения расстояния между многомерными временными рядами с помощью алгоритма DTW и его оптимизаций. Tраектория точки в 3х-мерном пространстве является примером многомерного временного ряда. При решении задачи нахождения расстояний исследуется наиболее оптимальный вид функции расстояния между двумя имерениями временного ряда. Поскольку нет полного обзора и исследований методов работы с временными рядами высокой размерности, необходимо модифицировать алгоритм и произвести исследование зависимости качества решения от расстояний между измерениями. В качестве прикладной задачи исследуется кластеризация данных мозговой активности обезьян. Качество кластеризации на основе полученной функции расстояния сравнивается с кластеризацией, полученной на основе авторегрессионной модели.

        \bigskip
        \textbf{Ключевые слова}: \emph {временные ряды, многомерные временные ряды, DTW, авторегрессионная модель}.
    }
    
    % \titleEng
    %     {JMLDA paper example: file jmlda-example.tex}
    % \authorEng
    %     {Author~F.\,S.$^1$, CoAuthor~F.\,S.$^2$, Name~F.\,S.$^2$}
    % \organizationEng
    %     {$^1$Organization; $^2$Organization}
    % \abstractEng
    %     {This document is an example of paper prepared with \LaTeXe\
    %     typesetting system and style file \texttt{jmlda.sty}.
    
    %     \bigskip
    %     \textbf{Keywords}: \emph{keyword, keyword, more keywords}.}
        
    \begin{document}

    \maketitle
    \section{Введение}
				
				В ряде задач возникает проблема анализа данных временных рядов. Для нахождения их сходства применяется функция расстояния, однако стандартный поточечный подход не является информативным \cite{01f4ab11a9ff49ff909094a135dcfe33}. Одним из способов решения этой проблемы является выравнивание временных рядов (DTW)  \cite{Keogh01derivativedynamic} и его модификаций \cite{journals/ida/SalvadorC07}. В работе рассматривается обобщение DTW на случай многомерных временных рядов. В работах \cite{Holt2007}, \cite{Sanguansat2012MultipleMS} предлагаются подходы к решеню этой задачи. В работе \cite{Holt2007} предлагается способ выравнивания многомерных рядов, основанный на нормализации исходных данных и нахождениии векторной нормы. В работе \cite{Sanguansat2012MultipleMS} рассматривается алгоритм, позволяющий выполнить выравнивание временных рядов между координатами. В данной работе исследуется возможность обобщения DTW для выравнивания многомерных временных рядов относительно общей временной шкалы, а также рассматривается возможность выравнивания значений координат друг относительно друга.
				
				Особенность этой работы заключается в использовании различных фукций расстояния для определения их влияния на точность выравнивания. Используются такие меры как $L_p$, RBF, косинусное расстояние и прочие.
				
				Перед работой над сравнением временных рядов рекомендуется провести их предварительную кластеризацию. Качество полученной полученной функции расстояния будет оцениваться как качество кластеризации временных рядов метрическими методами.
				
				С целью оценки качества полученных расстояний предлагается сравнивать качество кластеризации с кластеризацией на основании коэффициентов авторегрессионой модели для различных функций расстояния между выравненными рядами.
				
				Расстояния между выравненными рядами получаются как сумма поточечных расстояний и используются для кластеризации методом $k$\--средних.
				
				Данными для задачи служат данные мозговой активности обезьян и её зависимости от положения глаз. Данные представляют собой электрокортикограмму - зависимости потенциалов в 128 точках мозга и положения глаз от времени. В работе рассматривается кластеризация данных мозговой активности, что соответствует кластеризация 128-мерных временных рядов.
				
				
				
    \section{Постановка задачи}
				Dynamic time warping - измерение расстояния между двумя временными рядами.
				
				Задано два временных ряда, Q длины n и C длины m.
				
				$Q=q_1,q_2, ..., q_i, ..., q_n $
				
				$C=c_1,c_2, ..., c_j, ..., c_m $
				
				Требуется построить матрицу размера $n\times m$ c элементами $D_i_j=d(q_i, c_j)$, где d - выбранная метрика.
				
				Чтобы найти наибольшее соответсвие между рядами нужно найти выравнивающий путь W, который минимизирует расстояние между ними. W - набор смежных элементов матрицы D, $w_k = (i, j)_k$.
				
				$W = w_1,w_2, ..., w_k, ..., w_K $

				$max(n, m)\leq K \leq m+n+1$, где K-длина выравнивающего пути
				
				Выравнивающий путь должен удовлетворять следующим условиям:
				
				$w_1=(1,1)$, $w_K=(n, m)$
				
				$w_k = (a, b)$, $w_{k-1}=(a', b')$ : $a-a' \leq 1$, $b-b' \leq 1$ 
				
				$w_k = (a, b)$, $w_{k-1}=(a', b')$ : $a-a' \geq 0$, $b-b'\geq 0$
				
				Оптимальный выравнивающий путь должен минимизировать выравнивающую стоимость пути:
				
				$DTW(Q, C)=\displaystyle\sum_{k=1}^{K} w_k$
				
				Путь находится рекуррентно:
				
				$\gamma(i, j) = d(q_i, c_j) + min({\gamma(i-1, j-1), \gamma(i-1, j), \gamma(i, j-1)})$ , где $\gamma(i, j)$ суммарное расстояние, $d(q_i, c_j)$ расстояние в текущей клетке.
				
				
				
				
				
		
		\section{Название секции}
    \paragraph{Название параграфа.}
    \paragraph{Теоретическую часть работы}
    \section{Заключение}

		\bibliography{literature} 
    \bibliographystyle{unsrt}
    
    % Решение Программного Комитета:
    %\ACCEPTNOTE
    %\AMENDNOTE
    %\REJECTNOTE
    \end{document}
    

    
