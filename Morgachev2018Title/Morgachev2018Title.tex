\documentclass[12pt,twoside]{article}
    \usepackage{jmlda}
    %\NOREVIEWERNOTES
    \title
        {Динамическое выравнивание многомерных временных рядов}
    \author
        {Гончаров~А.\,В., Моргачев~Г.\,И., Смирнов~В.\,, Липницкая~Т.\,} % основной список авторов, выводимый в оглавление
    \thanks{
        Работа выполнена при финансовой поддержке РФФИ, проект \No\,00-00-00000.
        Научный руководитель:  Гончаров~А.\,В.
        Задачу поставил:  Гончаров~А.\,В.
        Консультант:  Гончаров~А.\,В.
    }
    \email
        {morgachev.gi@phystech.edu, smirnov.vs@phystech.edu, tanya.lipnizky@yandex.ru}
    \organization
        {МФТИ}

    \abstract{
        В работе рассматривается задача получения расстояния между многомерными временными рядами.
        На текущий момент для этого используется алгоритм DTW и его эффективные модификации.
        В классическом понимании DTW реализует способ получения расстояния между двумя одномерными рядами.
        Исследуется возможность его обобщения на многомерный случай.
        Рассматриваиются существующий подходы к решению данной задачи и исследуется зависимость качества выравнивания от использования
            различных функций расстояния.
        В качества прикладной задачи рассматривается кластеризация данных мозговой активности обезъян.
        Качество кластеризации на основе полученной функции расстояния сравнивается с алгоритмом на основе авторегрессионной модели.

        \bigskip
        \textbf{Ключевые слова}: \emph {Временные ряды, многомерные временные ряды, DTW}.
    }
    
    % \titleEng
    %     {JMLDA paper example: file jmlda-example.tex}
    % \authorEng
    %     {Author~F.\,S.$^1$, CoAuthor~F.\,S.$^2$, Name~F.\,S.$^2$}
    % \organizationEng
    %     {$^1$Organization; $^2$Organization}
    % \abstractEng
    %     {This document is an example of paper prepared with \LaTeXe\
    %     typesetting system and style file \texttt{jmlda.sty}.
    
    %     \bigskip
    %     \textbf{Keywords}: \emph{keyword, keyword, more keywords}.}
        
    \begin{document}

    \maketitle
    \section{Введение}

        В ряде задач анализа данных исследователям приходится иметь дело с измерениями, зависящими от времени. Последовательности такие измерений часто называют временными рядами.
        Для применения многих стандартных методов к подобным данным на них необходимо задать функцию расстояния.  
        Стандартный подход, базирующийся на сумме поточечным расстояний, плохо отражают схожесть временных рядов \cite{01f4ab11a9ff49ff909094a135dcfe33}.
        Одним из способов решения этой проблемы является выравнивание рядов друг относительно друга.
        \cite{Keogh01derivativedynamic} и его модификаций \cite{journals/ida/SalvadorC07}.
        В работе рассматривается обобщение DTW на случай многомерных рядов.
        
        В работах \cite{Holt2007}, \cite{Sanguansat2012MultipleMS} предлагаются подходы к решеню этой задачи.\
        В работе \cite{Holt2007} предлагается способ выравнивания многомерных рядов, основанный на нормализации исходных данных и нахождениии векторной нормы.
        В работе \cite{Sanguansat2012MultipleMS} рассматривается алгоритм, позволяющий выполнить выравнивание временных рядов между координатами.
        
        Предлагается рассмотреть влияниее различных функций расстояния на точность выравнивания: расстояние Минковского при $p = 1, 2$, косинусная мера.
        
        Зачастую, перед работой с временными рядами полезно провести их предварительную кластеризацию. Качество полученной функции расстояния будет оцениваться как качество кластеризации временных рядов метрическими методами.
        
        В качестве данных для задачи выступают данные мозговой активности обезъян и ее зависимости от положения глаз. Данные представляют собой электрокортикограмму - зависимости потенциалов в 128 точках мозга и положения глаз от времени. Рассматривается кластеризация данных мозговой активности, что соответствует кластеризация 128-мерных временных рядов.
        
        Исследуется возможность обобщения DTW для выравнивания многомерных временных рядов относительно общей временной шкалы.
        Также рассматривается возможность выравнивания значний координат друг относительно друга.
        
        Расстояния между выравненными рядами получается как сумма поточечных расстояний и используется для кластеризация методом $k$\- средних.
        Для оценки качества полученных расстояний предлагается сравнивать качества кластеризация с кластеризация на основании коэффициентов авторегрессионой модели для различных функций расстояния между выравненными рядами.
        

    \section{Название раздела}
    \paragraph{Название параграфа.}
    \paragraph{Теоретическую часть работы}
    \section{Заключение}

    \bibliography{literature} 
    \bibliographystyle{unsrt}
    
    % Решение Программного Комитета:
    %\ACCEPTNOTE
    %\AMENDNOTE
    %\REJECTNOTE
    \end{document}
    
